\documentclass{amu_these}

\input{symbols.tex}

\begin{document}

										%% page de titre
	\chead{}
	\pdfbookmark[0]{Page de titre}{titre}
	\thispagestyle{empty}
	\input{tex/titre.tex}
										%% licence
	\newpage
	\thispagestyle{empty}
	\input{tex/licence}


	\chapter*{Résumé}					%% résumé
	\input{tex/resume}
	\addcontentsline{toc}{chapter}{Résumé}

	\chapter*{Abstract}					%% abstract
	\lipsum[1]\index{Lorem ipsum}

\vspace{0.5cm}
Keywords: computational geometry, planar and rectangular complex, geodesic, global nonpositive curvature

\selectlanguage{french}

\lipsum[2]

\selectlangage{english}

	\addcontentsline{toc}{chapter}{Abstract}

	\chapter*{Remerciements}			%% remerciements
	\input{tex/remercie}
	\addcontentsline{toc}{chapter}{Remerciements}


    \microtypesetup{protrusion=false}	%% désactive la protrusion (TOC LOFT GLS)
	\tableofcontents					%% TOC
	\listoffigures						%% LOF
	\listoftables						%% LOT
	\printglossary[						%% Acronymes
		type=\acronymtype,
		title={Liste des acronymes},
		toctitle={Liste des acronymes}
		]
	\printglossary[						%% Glossaire
		title={Glossaire},
		toctitle={Glossaire}
		]
	\printglossary[						%% Nomenclature
		type=notation,
		title={Nomenclature},
		toctitle={Nomenclature}
		]
    \microtypesetup{protrusion=true}	%% rétabli la protrusion

	\ohead{\leftmark\ifstr{\rightmark}{\leftmark}{}{ -- \rightmark}}	%% place le chapître et la partie en en-tête

	%\chapter*{Introduction}
	\input{tex/intro}
	\addcontentsline{toc}{chapter}{Introduction}

	%\chapter{Généralités}
	\input{tex/chap1}

	\input{tex/chap2}
	
	\input{tex/chap3}

	%\chapter*{Conclusion}
	\input{tex/conc}
	\addcontentsline{toc}{chapter}{Conclusion}

	\appendix

	\newpage
	\printbibliography[					%% bibliographie
	heading=bibintoc
	]	
	\newpage
	\printindex							%% index
	\newpage
	\printendnotes						%% notes

%% annexes
	\setcounter{chapter}{0}
	\renewcommand{\thesection}{\Alph{section}}
	
	\chapter*{ANNEXES}
	\newpage
	\addcontentsline{toc}{chapter}{ANNEXES}
	
	\input{tex/annexe1}
	\newpage
	\input{tex/annexe2}

\end{document}
